\documentclass[a4paper]{article}

\usepackage{listings}
\usepackage{graphicx}
\usepackage[pdftex]{hyperref}

\newcommand\hex[1]{\hbox{\rm\H{}\tt#1}} % hexadecimal constant

\begin{document}
\title{The PowerTab file format}
\date{2004}
\author{Jelmer Vernooij}
\maketitle

Relevant links:

\begin{itemize}
\item \url{http://jelmer.vernstok.nl/oss/ptabtools/}
\item \url{http://www.power-tab.net/}
\end{itemize}

\begin{abstract}
This document describes the format of the PowerTab format, as 
used by the PowerTab editor. All information 
in this document was obtained by analysing data in existing PowerTab files.

This document should be used by those interested in adding support 
for the PowerTab file format to their own guitar tablature editors.
\end{abstract}

\section{Introduction}

Most of the programs I use run on Linux or other Unices. However, 
at the moment I still need to run guitar tablature editors on Windows, 
because most tabs are only available in proprietary file formats from 
programs such as Guitar Pro 3 and PowerTab.

This is why I decided to try to figure out the structure of the PowerTab 
format so I could write a small utility for generating Lilypond files 
from PowerTab files.

This document is the result of what I have figured out about the PowerTab 
format so far --- it's almost complete. The conversion utilities 
(published under the GPL) can be found on the website mentioned on the top of 
this page.

\section{General layout}

Every file has 4 parts: a header, a part containing the guitar tabs, a 
part containing the bass parts and a footer.

Each instrument (guitar or bass) contains ``elements''. Elements themselve can 
(at specified places) contain lists of elements. So we get a tree. 
There are several types of elements.

An instrument always contains 8 lists of elements, always in the same order.
They are:

\begin{enumerate}
\item CGuitar
\item CChordDiagram
\item CFloatingText
\item CGuitarIn
\item CTempoMarker
\item CDynamic
\item CSectionSymbol
\item CSection
\end{enumerate}

\section{General techniques}

\subsection{Lists}

\subsection{Strings}

Strings are always prepended by one byte describing their length. There is 
no ending null character.

\subsection{Item types}

\section{Header}

\subsection{Song}

\subsubsection{Public Release: Audio}
\subsubsection{Public Release: Video}
\subsubsection{Bootleg}
\subsubsection{Unreleased}

\subsection{Lesson}

\section{Types}

\section{Reference}

\subsection{CGuitar}

\subsection{CFloatingText}
\subsection{CChordDiagram}
\subsection{CTempoMarker}
\subsection{CLineData}
\subsection{CChordText}
\subsection{CGuitarIn}
\subsection{CStaff}
\subsection{CPosition}
\subsection{CSection}
\subsection{CDynamic}
\subsection{CSectionSymbol}
\subsection{CMusicBar}
\subsection{CRhythmSlash}
\subsection{CDirection}

\end{document}
